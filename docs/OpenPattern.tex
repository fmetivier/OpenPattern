\documentclass[11pt,a4paper]{article}
\usepackage[utf8]{inputenc}
\usepackage[francais]{babel}
\usepackage[T1]{fontenc}
\usepackage{amsmath}
\usepackage{amsfonts}
\usepackage{amssymb}
\usepackage{graphicx}
\usepackage{minted}
\usepackage[left=2cm,right=2cm,top=2cm,bottom=2cm]{geometry}
\usepackage{natbib}

\def\bl{\begin{minted}[texcl,mathescape]{sql}}
\def\el{\end{minted}}

\begin{document}
\title{OpenPattern}
\author{F. Métivier}
\date{\today}
\maketitle

\tableofcontents

\section{TODO}

\subsection{20/12}


Measurements:\\
- translate measurement names to english n\\
- measurement input procedure\\

Drawing:\\
- Write drawing routines with lines and comments IN PROGRESS\\


Patterns:\\
- add darts IN PROGRESS\\
- Bodice D change names (not mandatory though)\\
- Add donnanno sleeve for m\\
- Add donnanno everything for w\\
- Add Gilewska skirt, trousers for w,m\\
- Add chiapetta... in a some far future\\
- Add all basic models.\\

GUI:\\
- Everything !!!\\


\section{Introduction}

OpenPattern  est une librairie python, orientée objet,  dont l'objectif est de permettre le tracé de patrons de vêtements à l'échelle 1:1. 
Les patrons peuvent être sur-mesures, à  partir d'un ensemble de mesures prises sur le corps de la personne dont vous voulez tracer un patron, ou
correspondre à des mesures de prêt à porter classique. 
 
Les sources que j'ai utilisé pour tracer les patrons proviennent d'ouvrage de patronnages classiques. Ces ouvrages, destinés au couturiers amateurs ou aux étudiants d'écoles de modes, décrivent les techniques de réalisation de ce que l'on nomme la coupe à plat c'est-à-dire  la manière de projeter et tracer sur un plan (le papier puis le tissus) une forme géométrique 3D (le vêtement fini). J'ai utilisé les ouvrages de quatre auteurs ou autrices principaux. 
En premier lieu j'ai utilisé quatre tomes la série Le modélisme de mode de Teresa Gilewska pour la femme et pour l'homme\cite{Gilewska1,Gilewska2,Gilewska4,Gilewska5}. J'ai aussi eu recours à deux tomes des ouvrages d'Antonnio Donnanno pour femme et homme également \cite{Donnanno2005,Donnanno2016}. Pour les enfants j'ai utilisé le tome correspondant de Jacqueline Chiapetta  \cite{Chiappetta1999}. Enfin pour l'homme j'ai aussi eu recours au livre de  Claire Wargnier sur le vestiaire masculin \cite{wargnier2012}. Ces ouvrages sont tous rédigés de façon similaire. Ils exposent, en procédant pas à pas, les techniques empiriques de tracé d'un body, d'un pantalon, d'une chemise etc...
OpenPattern reprend ces techniques en les adaptant quand cela se révèle nécessaire\footnote{Ces ouvrages présentent parfois des coquilles ou des imprécisions qui  sont probablement corrigées par leurs auteurs quand elles ou ils enseignent.}.

\section{Structure}

OpenPattern est organisée autour de classes qui héritent les unes des autres. Cette structure présente l'avantage de coller aux techniques de coupe. En effet une chemise, une veste ou un gilet sont des adaptations d'un body, une robe dérive de l'assemblage d'une jupe et d'un body, enfin les jupes et les pantalons dérivent de formes génériques dites de bases.

OpenPattern présente pour l'instant deux niveaux d'héritage.
\begin{itemize}
\item La classe parente Pattern : C'est la classe générale qui définit ce qu'est un patron et les méthodes permettant de le dessiner et de le manipuler
\item Les classes de bases dérivées de Pattern: Basic\_Bodice, Basic\_Trousers, Cuffs, Collars
\item Les classes dérivées des classes de base: shirt, pyjamas qui héritent de Basic\_Bodice ou de Basic\_Trousers. Normalement, in fine ce sont celles que les utilisateurs «verront» ou utiliseront	
\end{itemize}


\subsection{Classe Pattern}
La liste des méthodes de cette classe est donnée ci-dessous. ces méthodes sont de quatre types dont seuls les trois premiers sont implémentés.

\begin{enumerate}
\item Les méthodes liées aux mesures qui permettent de charger ou d'enregistrer les modifications des mesures. Le choix pour l'instant est fait d'enregistrer les mesures dans des fichiers json.
\item Les méthodes de calculs couramment effectués en patronnage, mesurer une distance, un angle, chercher le point milieu d'un segment ou encore tracer une courbe. Ici le choix est fait d'utiliser des b-spline pour approximer le sacro-saint pistolet (on reviendra sur ce choix guidé au départ par la simplicité).
\item Les méthodes de dessin avec matplotlib.
\item Les méthodes d'altération de patrons qui permettent de déplacer des points des patrons de base et de recalculer les courbes.
\end{enumerate}



\begin{minted}{python}
class Pattern:

	def __init__(self, pname="sophie", gender='w'):
			
	# Catégorie 1		
	def get_measurements(self, pname="sophie"):
		
	def save_measurements(self, ofname=None):

	# Catégorie 2
	def intersec_manches(self, A, B, C, theta):

	def segment_angle(self, A, B):

	def middle(self, A, B):

	def distance(self, A, B):

	def pistolet(self, points, kval, ax=None, kwargs = {'color':'blue','linestyle':'solid'}, tot=False):
		
	# Catégorie 3	
	def draw_pattern(self, dic_list, vertices_list):

	def paper_cut(self, fig, ax, name='patternA4', paper='A4'):	

	def print_info(self, ax, model=None):

	def segment(self, A, B, ax, kwargs={'color':'blue'}):

\end{minted}
\section{Mesures}

Les mesures sont enregistrées dans des fichiers json dans la répertoire measurements.

\begin{itemize}

\item Différences H/F
\subitem tour de poitrine vs tour de bassin: > chez les hommes, < chez les femmes
\subitem hauteur taille dos, taille devant: > chez les hommes, < chez les femmes

\item Intéressant de noter qu'en fait l'enfant garçon passe de l'un à l'autre à la puberté (cf Chiapetta)

\item Est-ce que ces différences sont à l'origine des différences de dessin ? probablement mais rien n'est expliqué (ou j'ai mal lu).
\end{itemize}


\section{Bodice}  

\subsection{Naming convention}

trying to get standard
\begin{itemize}
\item		first letter Caps [then minor]
\begin{itemize}
\item		W: waist
\item		Sl: sleeve
\item		B: bust
\item		H: height
\item		C: collar
\item		Sh: shoulder
\item		DSl: depth of sleeve
\item		Cp: Control Point
\item		Hi: Hip
\end{itemize}
		
\item		Last Letter
\begin{itemize}
\item		B: Back
\item		F: Front
\end{itemize}
		
\item		No numbering when on the fold line numbering when on the sleeve side
\end{itemize}


Not done for Donnanno yet

\subsection{Comments}

\paragraph{Mesures}
ça part dans tous les sens notamment pour les hommes... le plus gros écart sépare ceux qui mesurent la largeur des épaules et ceux qui mesurent la longueur des épaules. Quelques un mesurent les deux mais c'est plus rare. On notera que les mesures type varient d'un livre à l'autre. Pas toujours de mesure du tour de bras, du tour de jaret, du tour de cuisse.

\paragraph{Shoulder line} 

For G-style women the shoulder construction relies on fixed angles. For G-style men it relies on distances from the shoulder line as for D-style. Why  this change ? This raises an issue as, if one follows the instruction, the shoulder is longer then measurements (if they exist) and the front shoulder and back shoulder lines do not have the same length. In the case the shoulder length is given, the solution is then to ajust the shoulder length to the measured length ones the shoulder line has been set up (as suggestion by Donnanno).
For C-style teenagers (boys) the shoulder relies on two different angles for the back (22$^o$)a nd the front (25$^o$). 

added a test for existence of true shoulder length measurement. If so then the soulder length is ajusted to fit measurements.

les largeurs d'épaules correspondent chez G homme si on ne baisse pas l'épaule devant de 7cm mais de 5 (comme chez donnanno).

\paragraph{Armhole and collar}

à noter que, pour une fois, pour l'homme tout est un peu fait par dessus 
la jambe. Ok des chemises des pantalons et des costumes c'est pas 
folichon mais quand même !

Donnano:  pour les femmes le buste de base présente un problème d'ajustement car il ne 
fournit pas de points de contrôles. J'ai repris ceux de Gilweska. 
Pour l'hommes: pas de buste de base j'en au  créé un à partir de la chemise de base. 
relativement simple à faire. Par contre problème (toujours) pour les 
longueurs d'épaule. La largeur d'épaule est donnée mais pas la 
longueur (qui d'ailleurs n'est que très rarement donnée pour les 
hommes). or on demande la longueur d'épaule (j'adore le e.g. 17cm mis 
dans l'exemple dont on ne sait pas d'où il sort) 

Gilewska : pour le buste homme j'ai ajouté  deux points de contrôle 
en base de manche  afin d'assurer la platitude d'emmanchure. Sinon les 
spleen ne veulent pas faire comme le perroquet.


Chiappetta for teenagers :  (figure~\ref{fig:CB14}) only uses one carrure measurement the back one. For teenagers above 10 she just retrieves two cms from  carrure measured on the back for the front. The back collar needs a second control point near the fold line to ensure the flatness of the spline.

Pour l'enfant Chiapetta changer l'angle, la longueur des points de contrôle et la carrure devant. de 2 à 8 ans les angles d'épaule devant et derrière sont les même, les longueurs plus petites et la carrure devant et dos sont les mêmes.

L'emmanchure et l'encolure sont faites avec des splines de second ordre. J'ai fini par jouer sur le fit des splines et les points de contrôle. À développer.

Bon mais comme d'hab Chiapetta ça à l'air un peu ringard sur les bords mais ça marche tout seul. je pense que je vais investir dans 
les bouquins adultes notamment pour l'homme...


\paragraph{Manches}
Gilewska: Aucune indication pour le bas de manche de base il faut donc 
se débrouiller seul avec le tour de poignet... 

Les splines ici sont du troisième ordre car il y a un point d'inflexion. 

\begin{figure}[hbtp]
\centering
\includegraphics[width=0.8\linewidth]{../patterns/Gilewska_Bodice_withour_dart_36.pdf}
\caption{Gilweska style Women bodice with no dart. (size 36)}
\end{figure}

\begin{figure}[hbtp]
\centering
\includegraphics[width=0.8\linewidth]{../patterns/Donnanno_Bodice_withour_dart_36.pdf}
\caption{Donnanno style Women bodice with no dart. (size 36)}
\end{figure}

\begin{figure}[hbtp]
\centering
\includegraphics[width=0.8\linewidth]{../patterns/Gilewska_Bodice_withour_dart_M36.pdf}
\caption{Gilweska style Men bodice with no dart. (size 36)}
\end{figure}

\begin{figure}[hbtp]
\centering
\includegraphics[width=0.8\linewidth]{../patterns/Chiappetta_bodice_G14.pdf}
\caption{Chiappetta style Boy (14 years old) bodice with no dart.}
\label{fig:CB14}
\end{figure}

De façon générale je trouve que les bustes hommes ne sont pas très ressemblant aux dessins des livres et je suis dubitatif car le programme reproduit exactement les instruction sauf quand c'est problématique de façon évidente (genre les largeur d'épaules de Donnanno).


\section{Collars}
Styles available from Gilewska men: Officer and OnePiece (for one piece collar)

\begin{figure}
\begin{center}
\includegraphics[width=0.48\textwidth]{../patterns/collar_Gilewska_OnePiece_M44G_FullSize.pdf} 
\includegraphics[width=0.48\textwidth]{../patterns/collar_Gilewska_Officer_M44G_FullSize.pdf} 
\end{center}
\caption{Collar styles}
\end{figure}


\section{Cuffs}
Styles availabel from Gilewska men : Simple and French

\begin{figure}
\begin{center}
\includegraphics[width=0.48\textwidth]{../patterns/cuff_Gilewska_Simple_M44G_FullSize.pdf} 
\includegraphics[width=0.48\textwidth]{../patterns/cuff_Gilewska_French_M44G_FullSize.pdf} 
\end{center}
\caption{Cuff styles}
\end{figure}



\bibliographystyle{plain}
\bibliography{OpenPattern.bib}

\end{document}
